\documentclass[beamer]{standalone}
%\documentclass[draft]{beamer}
\usepackage{mypackages}
%\includeonlyframes{current}

\begin{document}
\newcommand{\ParityPic}{
    \begin{tikzpicture}
      \node (origin) at (0,0) {};
        \node (5) [podd, left=1pt of origin, tok-here=5] {5};
        \node (6) [peven, left=of 5, tok-here=6]  {6};
        \node (1) [podd, right=of origin, tok-here=1]  {1};
        \node (2) [peven, right=of 1, tok-here=2]  {2};
        \node (3) [peven, above=of 2, tok-here=3] {3};

        \draw (3) edge [post, loop, looseness=5] (3)
            (3) edge [pre] (2)
            (3) edge [post, bend right] (6)
            (2) edge [post, bend left] (1)
            (2) edge [pre, bend right] (1)
            (2) edge [pre, bend left=63] (5)
            (1) edge [post] (5)
            (6) edge [pre,bend left] (5)
            (6) edge [post, bend right] (5);

        \node (graph) [fit=(1) (2) (3) (6) (5)] {};
        \node (odd-sym) [podd, above=1cm of 6, label={right:Odd}] {};
        \node [peven, above=7pt of odd-sym, label={right:Even}] {};
    \end{tikzpicture}
}
\newcommand{\ParityTree}{
\begin{tikzpicture} [
    every child/.style={norm},
    every node/.style={draw=black,thin},
    demph/.style={visible on=<-.(5)>},
    empht/.style={eet=<.(5)->2},
    emph/.style={ee=<.(5)->2},
    norm/.style={edge from parent/.append style={draw,black,thin,->}},
    oldn/.style={visible on=<{.(2),.(3)}>},
    oldc/.style={alt=<{.(2),.(3)}>{}{missing},every node/.append style=oldn},
    newstyle/.style={visible on=<{.(1),.(2),.(4)-}>},
]
    \node[podd, tok-here=1] (1) {1}
            child {
                node[podd, tok-here=5l] (l5) {5}
                child {
                    node[peven, tok-here=6ll] (l6) {6}
                    child[oldc] { node[podd, ncolor=2] {5} }
                    child[missing] { node {} }
                }
                child {
                    node[peven, tok-here=2l] (l2) {2}
                    child[empht] {
                        node[peven, tok-here=3l] (l3) {3}
                        child[empht] {
                            node[peven, tok-here=6lr] (l6') {6}
                            child[oldc] {node[podd,ncolor=2] {5}}
                        }
                        child[oldc] { node[peven,ncolor=1] {3} }
                    }
                    child[oldc] { node[podd,ncolor=1] {1} }
                }
            }
            child {
                node[peven, tok-here=2r] (r2) {2}
                child[oldc] { node[podd,ncolor=2] {1} }
                child[demph,every child/.append style={edge from parent/.append style=demph}] {
                    node[peven] (r3) {3}
                    child[oldc] { node[peven, ncolor=1] {3} }
                    child {
                        node [peven] (r6) {6}
                        child {
                            node [podd] (r5) {5}
                            child[oldc] { node[peven, ncolor=2] {2} }
                            child[oldc] { node[peven, ncolor=2] {6} }
                        }
                    }
                }
            };
        \begin{scope}[newstyle]
            \path[dashed, emph, color2,->] (r2) edge [ bend left=50] (1)
                  (l6) edge [ out=230, in=-180, looseness=2] (l5)
                  (l6') edge [ out=-120, in=-90] (l5);
            \path[dashed, demph, ->]
                  (r3) edge [color1, loop, in=180, out=240, looseness=9] (r3)
                  (l3) edge [color1, loop, in=0, out=-70, looseness=7] (l3)
                  (r5) edge [color2, bend left, out=90] (r2)
                  (r5) edge [color2, bend right, out=-120] (r6)
                  (l2) edge [color1, out=-60, in=-120, looseness=1] (1);
        \end{scope}
          %\begin{scope}[inner sep=2pt, minimum size=2pt, draw=none]
          %\node (odd-sym) [demph, right=of r2] {Odd Wins};
          %\node [above=2mm of odd-sym] {Even Wins};
          %\end{scope}
          \path (1) ++(3,0) coordinate (labelbot) ++(2em,0) coordinate (labeltextbot);
          \path (labelbot) ++(0,+1.5em) coordinate (labeltop) ++(2em,0) coordinate (labeltexttop);
          \draw[dashed, thick, color1, demph, newstyle] (labeltop) -- (labeltexttop);
          \draw[dashed, thick, color2, newstyle, newstyle] (labelbot) -- (labeltextbot);
          \node[right,draw=none,black, demph, anchor=west] at (labeltexttop) {Odd Wins} ;
          \node[right,draw=none,black,anchor=west] at (labeltextbot) {Even Wins} ;
          \draw[oldn] node[ncolor=1] at (labeltop) {} (labelbot) node[ncolor=2] {};
\end{tikzpicture}
}


\newcommand{\ShowParityPath}[5]{
    \ShowJustParityPath{#1}{#2}{%
            $\inf(\pi_#1)=\set{#3}$ \quad $\max \inf(\pi_#1) = #4$\\
            \alert{#5 wins}%
    }%
}

\newcommand{\ShowJustParityPath}[3]{
    \begin{overlayarea}{\textwidth}{4em}
        \only<+->{%
            $\pi_{#1}=$ \animatepath{#2} \only<.->{$\ldots$}
        \only<.->{%
            #3%
        }%
    }%
    \end{overlayarea}%
}

\begin{standaloneframe}[t]{Parity Games}
    \splitCol{0.5}{0.45}{
        \begin{block}{Winning conditions}<visible@+->
            \bigskip
            \ShowParityPath{1}{1,5,2,1,2,1,2}{1,2}{2}{Even}
            \bigskip
            \ShowParityPath{2}{1,5,2,1,5,2}{1,2,5}{5}{Odd}
            \bigskip
            \ShowJustParityPath{}{1,2,3,3,6,5,2,1}{\alert{$\text{Parity}(\max \inf (\pi))$ wins}}
        \end{block}
    }{\scalebox{0.77}{\ParityPic}}
\end{standaloneframe}

\begin{standaloneframe}{Parity Games}
    \begin{block}{Questions}
        \begin{itemize}
            \item Does either Even or Odd have a strategy to always win?\\
                \visible<2->{\alert{Yes}}
            \item If so, then how to compute the winning strategy?
            \item<visible@2-> Reduce to finite duration games
        \end{itemize}
    \end{block}
\end{standaloneframe}

\begin{standaloneframe}{Parity Games}
        \splitCol{0.4}{0.55}{%
            \medskip
            \scalebox{0.7}{\ParityPic}
            \begin{block}{Finite game}<visible@5->
                \begin{overprint}
                    \onslide<5> Even has a winning strategy
                    \onslide<6-> Every cycle has max priority even
                \end{overprint}
            \end{block}%
        }{\scalebox{0.7}{\ParityTree}}
            \addtocounter{beamerpauses}{7}%
            \splitCol{0.50}{0.45}{%
            \begin{overlayarea}{0.97\linewidth}{2.2cm}
            \begin{block}{Extension to infinite plays}<visible@.->
                \visible<.(1)->{$\pi=$} \enskip \animateStackedPath{1, 2r, 1, 5l, 6ll, 5l, 2l, 3l, 6lr, 5l} \enskip \only<.->{$\ldots$}
            \end{block}
            \end{overlayarea}
            }{%
                    \begin{itemize}
                        \item<.(-6)-> Every eliminated cycle has max priority even
                        \item<+-> Hence $\max\inf$ priority in $\pi$ is Even
                    \end{itemize}%
            }
\end{standaloneframe}

\begin{standaloneframe}{Parity Games}{Better Algorithms}
    \begin{itemize}[<+->]
        \item \fullcite{localstrat}\\
            \begin{center}
            Upper bound \footcite[see also][]{ediss13294} : $O \left( (n/d)^d \right)$
            \end{center}
        \item \fullcite{subexp}
            \[  n^{O(\sqrt{n})} \]
    \end{itemize}
\end{standaloneframe}
\end{document}
