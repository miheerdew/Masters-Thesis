\documentclass[beamer]{standalone}
\usepackage{mypackages}

\begin{document}
\begin{standaloneframe}[t]
    \frametitle{Parity Games}
    \begin{columns}
        \begin{column}{0.4\textwidth}
        \end{column}
        \begin{column}{0.6\textwidth}
            \scalebox{0.80}{
            \begin{tikzpicture}[podd/.style=p1, peven/.style=p2]
                  \node (origin) at (0,0) {};
                    \node (5) [podd, left=of origin, tok-here=5] {5};
                    \node (6) [peven, left=of 5, tok-here=6]  {6};
                    \node (1) [podd, right=of origin, tok-here=1]  {1};
                    \node (2) [peven, right=of 1, tok-here=2]  {2};
                    \node (3) [peven, above=of 2, tok-here=3] {3};

                    \draw (3) edge [post, loop, looseness=5] (3)
                        (3) edge [pre] (2)
                        (3) edge [post, bend right] (6)
                        (2) edge [post, bend left] (1)
                        (2) edge [pre, bend right] (1)
                        (2) edge [pre, bend left=63] (5)
                        (1) edge [post] (5)
                        (6) edge [pre,bend left] (5)
                        (6) edge [post, bend right] (5);

                    \node (graph) [fit=(1) (2) (3) (6) (5)] {};
                    \node (odd-sym) [podd, below=2cm of 6, label={right:Odd}] {};
                        \node [peven, below=7pt of odd-sym, label={right:Even}] {};
            \end{tikzpicture}
        }

            \foreach \x [count=\xi] in {1,5,2,1,2,3,3,\ldots} {
            \only<+-> {
                \global\let\TokenIsIn=\x
                \x
            }
        }
        \end{column}
\end{columns}
\end{standaloneframe}
\end{document}
