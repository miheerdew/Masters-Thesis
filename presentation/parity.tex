\documentclass[beamer]{standalone}
\usepackage{mypackages}

\begin{document}

\newcommand{\drawParityPic}{
                  \node (origin) at (0,0) {};
                    \node (5) [podd, left=of origin, tok-here=5] {5};
                    \node (6) [peven, left=of 5, tok-here=6]  {6};
                    \node (1) [podd, right=of origin, tok-here=1]  {1};
                    \node (2) [peven, right=of 1, tok-here=2]  {2};
                    \node (3) [peven, above=of 2, tok-here=3] {3};

                    \draw (3) edge [post, loop, looseness=5] (3)
                        (3) edge [pre] (2)
                        (3) edge [post, bend right] (6)
                        (2) edge [post, bend left] (1)
                        (2) edge [pre, bend right] (1)
                        (2) edge [pre, bend left=63] (5)
                        (1) edge [post] (5)
                        (6) edge [pre,bend left] (5)
                        (6) edge [post, bend right] (5);

                    \node (graph) [fit=(1) (2) (3) (6) (5)] {};
                    \node (odd-sym) [podd, above=1cm of 6, label={right:Odd}] {};
                    \node [peven, above=7pt of odd-sym, label={right:Even}] {};
}

\newcommand{\ParityPic}{
    \begin{tikzpicture}
        \drawParityPic
    \end{tikzpicture}
}

\newcommand{\ParityTree}{
\begin{tikzpicture} [
    every child/.style={norm},
    every node/.style={draw=black,thin},
    demph/.style={visible on=<-.(2)>},
    empht/.style={eet=<.(2)->2},
    emph/.style={ee=<.(2)->2},
    norm/.style={edge from parent/.style={draw,black,thin,->}},
]
    \node[podd, tok-here=1] (1) {1}
			child {
				node[podd, tok-here=5l] (l5) {5}
				child { node[peven, tok-here=6ll] (l6) {6} }
				child {
					node[peven, tok-here=2l] (l2) {2}
                    child[empht] { node[peven, tok-here=3l] (l3) {3}
                    child[empht] { node[peven, tok-here=6lr] (l6') {6} }
					}
					child[missing] { node {} }
					}
			}
			child {
				node[peven, tok-here=2r] (r2) {2}
				child[missing] { node {}
				}
                child[demph,every child/.append style={edge from parent/.append style=demph}] {
					node[peven] (r3) {3}
					child[missing] { node {}
					}
					child { node [peven] (r6) {6}
						child { node [podd] (r5) {5}
						}
					}
				}
			};
		\path[dashed, emph, color2,->] (r2) edge [ bend left=50] (1)
			  (l6) edge [ out=230, in=-180, looseness=2] (l5)
			  (l6') edge [ out=-120, in=-90] (l5);
        \path[dashed, demph, ->]
			  (r3) edge [color1, loop, in=180, out=240, looseness=9] (r3)
			  (l3) edge [color1, loop, in=0, out=-70, looseness=9] (l3)
			  (r5) edge [color2, bend left, out=90] (r2)
			  (r5) edge [color2, bend right, out=-120] (r6)
			  (l2) edge [color1, out=-60, in=-120, looseness=1] (1);

      %\begin{scope}[inner sep=2pt, minimum size=2pt, draw=none]
      %\node (odd-sym) [demph, right=of r2] {Odd Wins};
      %\node [above=2mm of odd-sym] {Even Wins};
      %\end{scope}
              \path (1) ++(3,0) coordinate (label);
              \draw[dashed, thick, color1, demph] (label) -- ++(2em,0) node[right,draw=none,black, demph] {Odd Wins} ;
              \draw[dashed, thick, color2] (label) ++(0,+1.5 em)  -- ++(2em,0) node[right,draw=none,black] {Even Wins} ;
\end{tikzpicture}
}

\DeclareDocumentCommand{\splitCol}{ O{0.4} O{0.6} m m}{
    \begin{columns}
        \begin{column}{#1\textwidth}
            #3
        \end{column}
        \begin{column}{#2\textwidth}
            #4
        \end{column}
    \end{columns}
}

\newcommand{\animatepath}[1]{
    \foreach \x/\y in {#1} {
            \only<.>{\settok{\y}}
            \uncover<+->{\x}
    }
}

%\begin{standaloneframe}[t]{Parity Games}
    %\fbox{
        %\scalebox{0.70}{\ParityPic}
    %} \bigskip

        %\pause
        %$\pi_1=$ \animatepath{1/1,5/5,2/2,1/1,2/2,1/1,$\ldots$/}{}

        %\visible<.->{
                %$\inf(\pi_1)=\set{1,2}$ \quad $\max \inf(\pi_1) = 2$\\
                %\alert{Even wins}
        %} \bigskip

        %\only<.>{\settok{2}}
        %\only<+->{
            %$\pi_2=$ \enskip 1 \enskip 5 \enskip 2 \animatepath{1/1,5/5,2/2,1/1,5/5, $\ldots$/}

            %\visible<.->{
                    %$\inf(\pi_2)=\set{1,2,5}$ \quad $\max \inf(\pi_2) = 5$\\
                    %\alert{Odd wins}
            %}
        %} \bigskip

        %\only<+->{
            %$\pi=$ \animatepath{1/1,2/2,3/3,3/3,6/6,5/5,2/2,1/1,$\ldots$/}
            %\visible<.->{
                %\alert{Parity $\max \inf (\pi)$ wins.}
            %}
        %}
%\end{standaloneframe}
\begin{standaloneframe}[t]{Finite Game}
    %\splitCol[0.7][0.3]{
    \begin{figure}
        \hspace*{-1cm}\scalebox{0.7}{\ParityTree}
        \hspace*{-1cm}\scalebox{0.7}{\ParityPic}
    \end{figure}
    %}{
        \only<.(2)>{Even has a winning strategy}
        \only<.(3)>{Every cycle has max priority even}
        \only<.(4)->{
            \addtocounter{beamerpauses}{4}
            $\pi=$\animatepath{1/1, 5/5l, 6/6ll, 5/5l, 2/2l, 3/3l, 6/6lr, 5/5l}
        }
    %}
\end{standaloneframe}
\end{document}
