\documentclass[beamer]{standalone}
\usepackage{mypackages}

\includeonlyframes{current}
\begin{document}
\newcommand{\MeanPayoffGame}{
\begin{tikzpicture}
      \node (origin) at (0,0) {};
      \node (a) [p2, tok-here=a, left=of origin] {a};
	  \node (b) [p2, tok-here=b, above right=of origin] {b};
	  \node (c) [p1, tok-here=c, below right=of origin] {c};

	  \draw[->]	(a) edge [post] node [below] {-2} (b)
				(a) edge [post] node [above] {-1} (c)
				(b) edge [post] node [left] {2} (c)
				(c) edge [post, bend right=60] node[left] {-2} (b)
				(c) edge [post, bend left=60] node [below] {-1} (a)
				(b) edge [post, bend right=60] node [above] {3} (a);

      \begin{scope}[inner sep=2pt, minimum size=1pt]
        \node (box) [p2, color2, right=1cm of c, label={right:Min}] {};
        \node (circ) [p1, color1, above=2mm of box, label={right:Max}] {};
      \end{scope}
  \end{tikzpicture}
}

\DeclareMyEvent{ShowP1}
\DeclareMyEvent{ShowOnlyP1}
\DeclareMyEvent{ShowP2}
\DeclareMyEvent{ShowOnlyP2}
\DeclareMyEvent{ShowOld}
\DeclareMyEvent*{ShowNew}

\newcommand{\MeanPayoffTree}{
    \begin{tikzpicture}[
    every child/.style={norm},
    every node/.style={draw=black,thin},
    norm/.style={edge from parent/.style={draw,black,thin,->}},
    %e1/.style={ee=<.(5)-.(6)>1},
    e1/.style={/myevent/on={ShowP1}{emph-edge=1},/myevent/on={ShowOnlyP1}{emph-edge=1}},
    e1t/.style={/myevent/on={ShowP1}{emph-edge-tree=1},/myevent/on={ShowOnlyP1}{emph-edge-tree=1}},
    %d1/.style={invisible on=<.(6)>},
    d1/.style={/myevent/on={ShowOnlyP1}{invisible}},
    d1t/.style={d1,every child/.append style={edge from parent/.append style=d1}},
    %e2/.style={ee=<.(7)->2},
    e2/.style={/myevent/on={ShowP2}{emph-edge=2},/myevent/on={ShowOnlyP2}{emph-edge=2}},
    %e2t/.style={eet=<.(7)->2},
    e2t/.style={/myevent/on={ShowP2}{emph-edge-tree=2},/myevent/on={ShowOnlyP2}{emph-edge-tree=2}},
    %d2/.style={invisible on=<.(8)->},
    d2/.style={/myevent/on={ShowOnlyP2}{invisible}},
    d2t/.style={d2,every child/.append style={edge from parent/.append style=d2}},
    %oldn/.style={visible on=<.(2)-.(3)>},
    oldn/.style={/myevent/off={ShowOld}{invisible}},
    %oldc/.style={alt=<.(2)-.(3)>{}{missing},every node/.append style=oldn},
    oldc/.style={/myevent/off={ShowOld}{missing}, every node/.append style=oldn},
    %values/.style={visible on=<.(3)>},
    %newstyle/.style={visible on=<{.(1),.(2),.(4)-}>},
    newstyle/.style={/myevent/off={ShowNew}{invisible}},
]

		\node[p2, tok-here=a] (a) {a}
            child[e2t] {
				node[p2, tok-here=bl] (bl) {b}
                child[e2t] {
                    node[p1, tok-here=cl] (cl) {c}
                    child[oldc] { node[p2] {-$\tfrac{1}{3}$} }
                    child[oldc] { node[p2] {0} }
                }
                child[oldc] { node[p2] {$\frac{1}{2}$} }
			}
            child[d2t] {
				node[p1,tok-here=cr] (cr) {c}
                child[oldc] { node[p2] {-1} }
                child[e1t] {
                    node [p2, tok-here=br] (br) {b}
                    child[oldc] { node[p2] {0} }
                    child[oldc] { node[p1] {0} }
                }
			};

            \path[dashed, ->, every node/.append style={draw=none,newstyle}, newstyle]
			(cl) edge[e1, out=-60, in=-90, looseness=2] node[right] {0} (bl)
			(cr) edge[d2, d1, out=45, in=0, looseness=1] node[above] {-1} (a)
            (cl) edge[out=225,in=160, looseness=2,d1] node[left] {-$\frac{1}{3}$} (a)
            (bl) edge[d2, out=-20,in=-90, looseness=1] node[below right, yshift=0.5em, pos=0.3] {$\frac{1}{2}$} (a)
			(br) edge[d2, out=-45, in=0, looseness=2] node[right] {0} (cr)
			(br) edge[d2, out=220, in=-80, looseness=2] node[pos=0.3,below left] {0} (a);
    \end{tikzpicture}
}

\newcommand<>{\s}[2]{\onslide#2{#1}}
\newcommand{\countFromHere}[1][+-]{
    \only<#1>{\number\counterDiff{beamerpauses}{overlaynumber}}}

\newcommand{\animateMeanPath}[3][1]{%
    \begin{overlayarea}{#1\textwidth}{2cm}
        \directlua{CmdAnimateMeanPath(\luastring{#2}, \luastringN{#3})}%
    \end{overlayarea}
}

\begin{standaloneframe}{Mean Payoff Games}
    %\newcommand{\ainmateFraction}[1]{
    %\luaexec{
        %local str=require 'std.string'
        %for i=#1,#2
    %}}
    \splitCol{0.55}{0.40}{
        \stepcounter{beamerpauses}
        \begin{block}{Payoffs}
            \begin{exampleblock}{$(ab)^\omega$ \onslide<.(13)->{Min pays $\frac{1}{2}$ units to Max}}<visible@+->
                \AnimateMyToken{a,b,a,b}
                \animateMeanPath{a/-2, b/+3, a/-2, b/+3, a/-2, b/+3, a/-2}{\sim \tfrac{n(-2+3)}{2n} \to \frac{1}{2}}
            \end{exampleblock}
            \begin{exampleblock}{$(acb)^\omega$ \onslide<.(15)->{Min pays $-\frac{1}{3}$ units to Max}}<visible@+->
                \AnimateMyToken{a,c,b,a,c,b}
                \animateMeanPath{a/-1, c/-2, b/+3, a/-1, c/-2, b/+3, a/-1}{\sim \tfrac{n(-1-2+3)}{3n} \to 0}
            \end{exampleblock}
        \end{block}

        \uncover<+->{
        \uncover<+->{Generally}
        \begin{itemize}
            \item Min tries to minimize \alt<.(-1)>{$\lim$}{$\limsup$}
            \item Max tries to maximize \alt<.(-1)>{$\lim$}{$\liminf$}
        \end{itemize}
        }
    }{\scalebox{0.8}{\MeanPayoffGame}}
\end{standaloneframe}

\begin{standaloneframe}{Mean Payoff Games}
    \begin{block}{Questions}
        \begin{itemize}
            \item Does the game have a value? i.e. is there a $v$ so that
                \begin{itemize}
                    \item Max can ensure $\liminf \geq v$
                    \item Min can ensure $\limsup \leq v$
                \end{itemize}
                \visible<2->{\alert{Yes}}
            \item<2-> How to compute the optimal strategies?
        \end{itemize}
    \end{block}
    \visible<3->{Solution using the finite game}
\end{standaloneframe}

\begin{standaloneframe}[label=current]{Mean Payoff}{Finite Game}
    \only<2>{\SetMyEvent{ShowOld}}
    \only<3>{\SetMyEvent{ShowOld} \SetMyEvent*{ShowNew}}
    \only<5>{\SetMyEvent{ShowP1}}
    \only<6>{\SetMyEvent{ShowOnlyP1}}
    \only<7>{\SetMyEvent{ShowP2}}
    \only<8-22>{\SetMyEvent{ShowOnlyP2}}
    \only<23>{\SetMyEvent{ShowOnlyP1}}
    \scalebox{0.7}{\MeanPayoffTree}
    \scalebox{0.7}{\MeanPayoffGame}

    \bigskip
    \begin{overlayarea}{\textwidth}{2ex}
        \only<6>{Max can ensure $\geq 0$ in the finite game}
        \only<8-22>{Min can ensure $\leq 0$ \onslide<9->{\alert<9>{in the mean payoff game too}}}
        \only<23>{Max can ensure $\geq 0$}
    \end{overlayarea}
    %\visible<.(6),.(8)-22>{ \alt<.(6)>{Max}{Min} can ensure \alt<.(6)>{$\geq$}{$\leq$} 0 \visible<.(9)->{\alert<.(9)>{in the mean payoff game too}}}
    \bigskip

    \addtocounter{beamerpauses}{9}
    \splitCol{0.45}{0.5}{
       \visible<10-22>{
       \begin{overlayarea}{0.95\textwidth}{5.5em}
           $\pi=$ \enskip \animateStackedPath{a,bl,cl,bl,cl,a,bl,cl,a}
        \end{overlayarea}
    }
    }{
        \begin{overprint}
            \only<.(-1)-.>{Every time a cycle with average value $\leq 0$ is eliminated}
            \only<+>{Hence limsup of averages of $\pi$ is $\leq 0$}
            \only<+->{
                \begin{itemize}
                    \item Similarly Max can ensure liminf of the average is $\geq 0$
                    \item Hence the value of Mean payoff game is 0
                \end{itemize}
            }
        \end{overprint}
            }
\end{standaloneframe}
\end{document}
