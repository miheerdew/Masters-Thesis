\chapter{Parity}

Parity games were introduced in \cite{emersonjutla} to solve the $\mu$-calculus model checking problem. From the view of Complexity Theory, the problem of deciding the winner of a parity game is in $\NP \cap \coNP$ (so unlikely to be $\NP$-complete), but the question of whether it has a polynomial time solution remains open.

Apart from Parity, the following chapters will also introduce Mean and Discounted Payoff Games. Although seemingly unrelated, each game can be reduced to the successive one. This is used in \cite{parityup} to show that the decision problems for all of them are in $\UP \cap \coUP$ (here $\UP$ is the class of decision problems that have a unique polynomial size certificate, hence $\P \subseteq \UP \subseteq \NP$). Whether any of these games (i.e. their respective decision problems) has a polynomial time solution or not is again an open question.

\section{Definition}
%TODO: Make this consitent with notation in Chapter 1
%Idea: Call it G instead of \G like done in Mean Payoff
Let $G=(V_0,V_1,E)$ be a graph and let (for some $M \in \Nat$)
\[
    p : V \mapsto \set{0,1,2 \ldots M}
\]
be an assignment of integer priorities to each vertex.

Parity game $\G_p=(G,Win_p)$ is a Win-Lose game (section \ref{sec:winlose}) with
\begin{align}
    Win_p &= \set{ \left(v_0 v_1 \ldots\right) \in \PI \mid \left(\limsup_{i \geq 0} p(v_i)\right)  \text{ is Odd }} \label{cond:parity}
\end{align}

Let the largest priority seen infinitely often, on a path $\pi$, be denoted by \mip\ of $\pi$. If \mip\ of $\pi$ is odd then \Player{1} wins $\pi$; if the \mip\ is even then \Player{0} wins. Since $V$ is finite, there's a vertex $v$ which is visited infinitely often by $\pi$, and has the same priority as the \mip.

Notice that in contrast with finite games (section \ref{sec:finitegames}), the winning condition \eqref{cond:parity} only depends on long run behavior. This property is also called prefix independence -- for $w \in \PF$ and $\alpha \in \PI$ if $w \cdot \alpha \in \PI$ then $\alpha \in Win_p \iff w \cdot \alpha \in Win_p$. However, due to path constraints in the graph G, the prefix might still be important as it can enable or disable certain long term behaviors.


\section{Positional Determinacy}
From every vertex $v \in V$ the parity game $(\G_p,v)$ is determined (section \ref{subsec:determinacy}) -- one can invoke the general determinacy theorem by Martin \cite{martin_borel_1975}  to show this. However a stronger result was proved by Emerson \cite{emerson_automata_1985} --\\


\begin{theorem}
    \label{thm:posParity}
    For any parity game $\G_p=(G,Win_p)$, there are positional strategies $(\sigma^*,\tau^*)$ and a partition of $V = W_0 \sqcup W_1$, so that $\sigma^*$ is a winning strategy for \Player{0} from each $v \in W_0$ and $\tau^*$ is a winning strategy for \Player{1} from each $v \in W_1$.
\end{theorem}
Hence, not only is the game $(\G_p,v)$ determined, the winner has a positional winning strategy independent of $v$ (in the player's winning region).

Given $v \in V$ and positional strategies $(\sigma,\tau)$, since $\pi^{v}_{\sigma\tau}$ is ultimately periodic we have 
\[
    \pi^{v}_{\sigma\tau} \in Win_p \iff \left( \max_{v \in \cycle(\pi^v_{\sigma\tau})} p(v)\right) \text{ is odd }
\]
Hence given $(\sigma,\tau)$, the winner of $\pi^v_{\sigma\tau}$ can easily be decided by analysing the cycle reachable in $G_{\sigma\tau}$.  Then by Theorem \ref{thm:posParity}, one could find the winner of $(\G_p,v)$ by enumerating over all possible positional strategies (which are finitely many) of both the players, and find one which beats all the strategies of the opponent. 

However something better can be done. Given a positional strategy $\sigma$, it is directly possible to check if this strategy is winning for \Player{0} in $(\G_p,v)$ or not. Look at the graph $G_\sigma$. $\sigma$ is winning in $(\G_p,v)$ if and only if no run in $G_\sigma$ starting at $v$ has an odd \mip\ -- this is the emptiness problem for parity word automata and can be solved in in $O\left((|V|+|E|) \log M\right)$ (see \cite{king_complexity_2001}). Hence to check if \Player{0} wins $(\G_p,v)$ one could guess a strategy $\sigma$ (a polynomial size certificate) and check in polynomial time whether it is winning for \Player{0} or not. Hence the decision problem of whether \Player{0} wins $(\G_p,v)$ is in $\NP$. This problem is also in $\coNP$ as one could apply the same procedure for \Player{1} (i.e. for ``no" instances, guess a strategy for \Player{1} and verify it). Hence we have the following theorem from \cite{thomas2002automata}\\
\begin{theorem}
    Given a graph $G$, priorities $p$, and $v \in V$. The decision problem of whether \Player{0} can win $(\G_p,v)$ is in $\NP \cap \coNP$
\end{theorem}

There are a couple of proofs of Theorem \ref{thm:posParity}, some of which are constructive. In \cite{subexp} the proof due to Zielonka \cite{zielonka_infinite_1998} and McNaughton \cite{mcnaughton_infinite_1993} is used to obtain sub-exponential algorithm $O(n^{\sqrt{n}})$ for deciding the winning regions. However whether a polynomial time algorithm exists is not known. 

In the following, we will provide a proof from \cite{bjorklund_memoryless_2004} of determinacy of Parity Games. We will not cover the existence of positional optimal strategies. As seen already, positional determinacy is an important property; it can be proven along the lines of the following proof using some more work (see \cite{bjorklund_memoryless_2004}).

\section{Finite Game}
\label{sec:parityFiniteGame}

Now we introduce a finite game $\G^f_p=(G,Win^f_p)$ closely related to the Parity Game $\G_p=(G,Win_p)$. It is played on the same graph $G$ but it stops as soon as a vertex repeats (which will happen within $|V|$ rounds). Assume that the resulting path is 
\[
    \pi = v_0v_1 \ldots v_r \ldots v_k v_{k+1}
\] 
with $v_r = v_{k+1}$ and all $v_i$'s distinct for $i \leq k$. Denote the unique cycle formed by $\cycle(\pi)=v_r v_{r+1} \ldots v_k$. If the largest priority in $\cycle(\pi)$ is odd then \Player{1} wins the play, otherwise \Player{0} wins. That is  
\[
    \pi \in Win^f_p \iff \left(\max_{v \in \cycle(\pi)} p(v)\right) \text{ is Odd }
\]

Notice that this is the same condition as the ultimately periodic path $v_0 \ldots v_{r-1} (v_r \ldots v_k)^\omega$ winning in $\G_p$. Hence $\G^f_p$ is the game that would result from $\G_p$ if both the players decided to play positional strategies. Since $\G^f_p$ is a finite game, by Corollary \ref{cor:finiteDeterminacy} starting at any $v \in V$ one of the players has a winning strategy in $(\G^f_p, v)$. Hindsight from Theorem \ref{thm:posParity} tells us that the same player will also be the winner of $(\G_p,v)$. We will show this without using Theorem \ref{thm:posParity} and hence prove determinacy of $(\G_p,v)$.\\
\begin{theorem}
    \label{thm:parityDet}
    Given a (possibly history dependent) winning strategy $\rho$ for \Player{i} in $(\G^f_p,v)$, there is a finite memory winning strategy $\tilde{\rho}$ for \Player{i} in $(\G_p,v)$.
\end{theorem}

The following stack based technique will be used to construct the $\tilde{\rho}$. This will be useful later too. Notice that $\rho$ is only defined (or relevant) for simple paths starting in $v$.

\subsection{Stack based extension} 
\label{sec:stack}
Let $\rho$ be a strategy for \Player{i} defined on simple paths. The following describes $ \tilde{\rho}$ -- an extension of $\rho$ to arbitrary plays using finite memory. At each stage \Player{i} maintains a simple path of $G$ in its memory. If the game starts at $v_0$, let $m_0=v_0$. At the \nth[k+1]\ stage, if it is \Player{i}'s turn, play $v_{k+1}=\rho(m_k)$ (otherwise the opponent decides $v_{k+1}$), and update the memory as follows.\\
Assume
\begin{align*}
    m_k &=u_0 u_1 \ldots u_r \quad r \geq 0\\
    \intertext{then}
    m_{k+1}&=\begin{cases}
        u_0 u_1 \ldots u_s & \text{ if } u_s = v_{k+1} \text{ for some } s \leq r \\
        u_0  u_1 \ldots u_r v_{k+1} & \text{ otherwise }
        \end{cases}
\end{align*}
In the first case $u_{s+1} \ldots u_r v_{k+1}$ forms a cycle in $G$, which had to be eliminated for $m_{k+1}$ to remain a simple path.

For each $n$, $m_n$ is a simple path from $v_0$ to $v_n$ which conforms with $\rho$. If $C_1,C_2 \ldots C_{r_n}$ are the cycles eliminated by the \nth\ round then $m_n, C_1, C_2 \ldots C_r$ forms a partition of $v_0v_1 \ldots v_n$


\begin{proof}[Proof of Theorem \ref{thm:parityDet}]
    Obtain $\tilde{\rho}$ from $\rho$ by the above stack based extension. We will show that this strategy $ \tilde{\rho}$ is winning for \Player{i} in $(\G_p,v)$.

    Assume $\rho$ is a strategy for \Player{0}. The proof for \Player{1} is similar. Let $\pi = v v_1 v_2 \ldots $ be an infinite path that conforms with $\tilde{\rho}$. We will show that \mip\ for $\pi$ is even. This shows that $\tilde{\rho}$ is winning for \Player{0} in $(\G_p,v)$.
    
    Since $\rho$ is winning for \Player{0} in $(\G^f_p,v)$, for any path that starts at $v$ and conforms with $\rho$ till the first cycle formed, the largest priority of the cycle will be even. In the stack based implementation of $ \tilde{\rho}$ on $\pi$, each $m_i$ conforms with $\rho$, hence the max priority in each of the eliminated cycles $C_r$ will be even. Given this, the \mip\ of $\pi$ cannot be odd. Indeed, let $u$ be a vertex which is visited infinitely often in $\pi$, and has same priority as \mip\ of $\pi$. Then $u$ must be a part of infinitely many $C_i$'s. But if $p(u)$ is odd, then there will be a vertex in each of these cycles which has priority strictly greater than $p(u)$ (since max priority in every $C_i$ is even). Since $C_i$'s correspond to disjoint positions of $\pi$, this shows that the \mip\ of $\pi$ is strictly greater than $p(u)$. A contradiction.

\end{proof}
