\chapter{Conclusion}

We have introduced Parity, Mean payoff and Discounted payoff games. All of these games fit into the framework presented in \Cref{chap:formalism} and have positional optimal strategies. In fact, \cite{pos_mixing} provides sufficient conditions (called fairly mixing property) on the payoff function $f$, under which the game $(G,f)$ will have positional optimal strategies, and shows that Parity, Mean and Discounted payoffs satisfy those conditions.

Both Parity and Mean payoff games have prefix independent payoffs. Discounted payoff (with discount parameter $\lambda$) is prefix dependent, but the dependence on the prefix decreases as $\lambda \to 1$. On the other hand for a fixed $\lambda < 1$, the discounted payoff can be determined to any desired accuracy by knowing a large enough prefix. As a result, for a fixed $\lambda$, finding the value for the discounted game $\G^\lambda_w$ is easy, but as $\lambda \to 1$ the Discounted payoff approximates the Mean payoff (\Cref{thm:mean-to-discounted}) and the problem becomes difficult.

The decision problem for Parity games \nameref{dec:parity} can be reduced in polynomial time to the decision problem for Mean payoff games \nameref{dec:meanpayoff} (\Cref{sec:parity-to-mean}). This was made possible by the reduction between their corresponding finite games (which are very similar). Using the relation between Mean and Discounted payoff games \nameref{dec:meanpayoff} can be reduced in polynomial time to the decision problem for Discounted payoff games \nameref{dec:disc} (\Cref{cor:mean2disc})

Hence in this order -- \nameref{dec:parity}, \nameref{dec:meanpayoff}, \nameref{dec:disc}, each problem is harder than the previous. By \Cref{thm:disc-in-up}, \nameref{dec:disc} (the hardest of them) is in $\UP \cap \coUP$ -- hence each of them is in $\UP \cap \coUP$. Whether any of them have a polynomial time solution or not, is not known. There have been many attempts at better algorithms for \nameref{dec:parity} (see \cite[Chap~7]{thomas2002automata}), but recently \cite{ediss13294} provided exponential lower bounds for some of the approaches which seemed promising.

From here one could look at Simple Stochastic Games (SSG) \cite{condon_stochastic}. \cite{zwick_meanpayoff} provides a reduction from \nameref{dec:disc} to SSG. Hence SSG are harder than all the games presented here, however they too are in $\UP \cap \coUP$ (and have positional strategies). \cite{parity2ssg} also provides a direct reduction from \nameref{dec:parity} to SSG.
