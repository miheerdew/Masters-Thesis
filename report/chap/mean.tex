\chapter{Mean payoff}

Mean payoff is another infinite duration game, closely related to parity games. It was studied independently in \cite{ehrenfeucht_positional_1979} and \cite{gurvich_cyclic_1988}.

\section{Definition}
Let $G=(V_0,V_1,E)$ be a graph and
\[
    w : E \mapsto \Int
\]
be edge weights. Assume $|w(e)| \leq W$ for every $e \in E$.

The Mean payoff game is an infinite duration game $\G_w=(G,f_w)$ where we want the payoff
\begin{equation}
    f_w (v_0v_1v_2 \ldots) = \lim_n \frac{1}{n} \sum_{i=0}^{n-1} w(v_i,v_{i+1}) \label{eq:meanpayoff}
\end{equation}
to be the mean weight on the infinite path. Unfortunately the limit in \eqref{eq:meanpayoff} may not always exist.
We can also look at $\gsup=(G,\overline{f_w})$ and $\ginf=(G,\underline{f_w})$, where $\fsup$, $\finf$ are obtained by replacing $\limsup$, $\liminf$ in place of $\lim$ in \eqref{eq:meanpayoff}. We will see that the choice will not matter for optimal play. %But to make things precise, the mean payoff game will be the game $\G_w=(G,\fsup,\finf)$ where \Player{0} tries to minimize $\fsup(\pi)$ and \Player{1} tries to maximize $\finf(\pi)$. This deviates slightly from our framework (\autoref{sec:formalism}) which is zero sum.

When the play $\pi$ is ultimately periodic the limit in \eqref{eq:meanpayoff} exists and is equal to $\mean(\cycle(\pi))$. Where
\begin{equation}
    \mean(v_1 v_2 \ldots v_k) = \frac{1}{k} \left( \sum_{i=1}^{k-1} w(v_i, v_{i+1}) + w(v_k, v_1) \right) \label{eq:mean}
\end{equation}
is the mean weight on the cycle $v_1 \ldots v_k \in \PF$ (with $(v_k,v_1) \in E$).
As a consequence when both \Player{0} and \Player{1} play finite memory starategies $(\sigma,\tau)$
\[
    f_w(\sigma,\tau) = \fsup(\sigma,\tau) = \finf(\sigma,\tau)
\]

Just like the parity winning condition both $\fsup,\finf$ are prefix independent (a payoff $f$ is prefix independent if $f(\alpha) = f(w \cdot \alpha)$ for any $\alpha, w \cdot \alpha \in \PI$).

\section{Optimal strategies}

The following theorem is the central result in \cite{ehrenfeucht_positional_1979}.

\begin{theorem}
    \label{thm:posMeanPayoff}
    There are positional strategies $(\sigma^*,\tau^*)$ for \Player{0}, \Player{1} respectively and a value vector $\eta \in \Reals^V$ so that
    \begin{equation}
        \label{eq:mpminmax}
        \begin{aligned}
            \fsup(\sigma^*,\tau) &\leq \eta \qquad \forall \tau \in \ST{1}\\
            \finf(\sigma,\tau^*) &\geq \eta \qquad \forall \sigma \in \ST{0}
        \end{aligned}
    \end{equation}
\end{theorem}

In other words starting from any $v$, \Player{0} can ensure that the $\limsup$ of the means remains below $\eta_v$, while \Player{1} can ensure that the $\liminf$ of the means remains above $\eta_v$. Call any $(\sigma^*,\tau^*)$ which satisfy \eqref{eq:mpminmax} optimal strategies for $\G_w$ and $\eta$ its value vector. When $(\sigma^*,\tau^*)$ is played, the limit in \eqref{eq:meanpayoff} exists and $f_w(\sigma^*,\tau^*)=\eta$

Since $\fsup \geq \finf$ this shows that $\eta$ is the minimax value for both $\gsup$ and $\ginf$ (hence $\eta$ is unique), and the strategies $(\sigma^*,\tau^*)$ are optimal in these games too. It will follow from the proof later that if $(\sigma,\tau)$ are finite memory optimal strategies for $\gsup$ (or $\ginf$) they will also be optimal for $\G_w$ (i.e. they will satisfy \eqref{eq:mpminmax}).\\

The following is a simple consequence of \autoref{thm:posMeanPayoff} which will be useful while approximating the value later.
\begin{corollary}
    \label{cor:valMeanPayoff}
    Let $\eta$ be the value of $\G_w$. Then for any $v \in V$,
  $\eta_v=\frac{n}{m}$ for some $n,m \in \Int$ with $|n| \leq W \cdot |V|$ and $1 \leq m \leq |V|$.
\end{corollary}
\begin{proof}
     Since $(\sigma^*,\tau^*)$ are positional
    \[
        \eta_v = f^v_w(\sigma^*,\tau^*) = \mean(\cycle(\pi^v_{\sigma^*\tau^*}))
    \]
    is the average weight over some simple cycle in $G$. Hence it has the required form.
\end{proof}
We will prove a weaker version of \autoref{thm:posMeanPayoff} using the same technique used for Parity.
\begin{theorem}
    \label{thm:meanEquilibrium}
    There are finite memory strategies $(\sigma^*,\tau^*)$ for \Player{0}, \Player{1} and a value vector $\eta \in \Reals^V$ so that
    \begin{equation*}
        \begin{aligned}
            \fsup(\sigma^*,\tau) &\leq \eta \qquad \forall \tau \in \ST{1}\\
            \finf(\sigma,\tau^*) &\geq \eta \qquad \forall \sigma \in \ST{0}
        \end{aligned}
    \end{equation*}
\end{theorem}
Just like in the case of parity, the complete proof of \autoref{thm:posMeanPayoff} can be obtained by doing some more work. In fact the presentation here, taken from \cite{bjorklund_memoryless_2004}, is the first part of a uniform proof for both \autoref{thm:posParity} and \autoref{thm:posMeanPayoff}.

\section{Finite Game}
Like in \autoref{sec:parityFiniteGame}, let us introduce a finite duration game $\G^f_w=(G,f'_w)$ related to $\G_w=(G,f_w)$. The game is played on $G$ and stops the first time a vertex repeats. The payoff associated with such a path is
\[
    f'_w(\pi) = \mean(\cycle(\pi))
\]
Where $\cycle(\pi)$ is the first cycle formed -- see \eqref{eq:cyclefinite}.

Notice that $\G^f_w$ is obtained from $\G_w$ when the players restrict to positional strategies. Since $\G^f_w$  is a finite game by \autoref{thm:finiteMinMax} there is a payoff vector $\eta \in \Reals^V$ and optimal strategies $(\sigma,\tau)$ in $\G^f_w$. Note that any for path starting at $v$ which conforms with $\sigma$ till the first cycle formed, the mean value of the cycle will be $\leq \eta_v$. Similarly, for path starting at $v$ which conforms with $\tau$ till the first cycle formed, the mean value of the cycle will be $\geq \eta_v$.  Now we will show that this $\eta$ is also the value vector for $\G_w$.

\begin{proof}[Proof of \autoref{thm:meanEquilibrium}]
    Let $\eta$ be the value vector and $(\sigma,\tau)$ be the optimal strategies for $\G^f_w$. Use the stack based technique (\autoref{sec:stack}) to obtain $\sigma^*$ from $\sigma$ and $\tau^*$ from $\tau$. We will now show that \eqref{eq:mpminmax} is satisfied.

    Let us show the first inequality. Take any path $\pi = v_0 v_1 \ldots$ which conforms with $\sigma^*$. Then in the stack based implementation of $\sigma^*$ on $\pi$, each $m_i$ is a path starting at $v_0$ which confirms with $\sigma$. Hence each of the eliminated cycles $C_i$ will have $\mean(C_i) \leq \eta_{v_0}$. If by the \nth\ stage $m_n=u_0 \ldots u_{s_n}$ and cycles $C_1,\ldots, C_{r_n}$ have been eliminated, then
    \begin{align*}
        \sum_{i=0}^{n-1} w(v_i,v_{i+1}) &= \sum_{i=0}^{s_n-1} w(u_i,u_{i+1}) + \sum_{j=1}^{r_n} \Sum(C_j)\\
        \intertext{where}
        \Sum(v_1v_2 \ldots v_k) &= \sum_{i=1}^{k-1} w(v_i,v_{i+1}) + w(v_k,v_1)
        \intertext{is the sum of the weights on the cycle. Hence if $|C|$ denotes the number of edges (or vertices) in $|C|$. We have}
        \sum_{i=0}^{n-1} w(v_i,v_{i+1}) &= \sum_{i=0}^{s_n-1} w(u_i,u_{i+1}) + \sum_{j=1}^{r_n} |C_j|\mean(C_j) \numberthis \label{eq:sumcycles}\\
        & \leq s_n W + \eta_{v_0} \sum_{j=1}^{r_n} |C_j| = s_n W + \eta_{v_0} ( n - s_n ) \\
        & \leq n \eta_{v_0} + s_n (W - \eta_{v_0})
        \intertext{since $s_n + \sum_{j=1}^{r_n} C_j = n$. Divide by $n$.}
        \frac{1}{n} \sum_{i=0}^{n-1} w(v_i,v_{i+1}) &\leq \eta_{v_0} + \frac{s_n}{n} ( W - \eta_{v_0})\\
        \intertext{Since $m_n$ is a simple path, $|s_n| \leq |V|$. Let $n \to \infty$}
        \limsup_n \frac{1}{n} \sum_{i=0}^{n-1} w(v_i,v_{i+1}) &\leq \limsup_n \left( \eta_{v_0} + \frac{s_n}{n} ( W - \eta_{v_0})\right) = \eta_{v_0}\\
        \intertext{This shows}
        \fsup(v_0 v_1 \ldots) & \leq \eta_{v_0}\\
        \intertext{Since $v_0 v_1 \ldots$ was any path that conformed with $\sigma^*$ we have}
        \fsup(\sigma^*,\tau) &\leq \eta \quad \forall \tau \in \ST{1}
    \end{align*}

    To show the second inequality of \eqref{eq:mpminmax} proceed similarly. Let $\pi=v_0 v_1 \ldots$ be a path the conforms with $\tau^*$. In the stack based implementation of $\tau^*$ on $\pi$ each of the eliminated cycles $C_i$ will have $\mean(C_i) \geq \eta_{v_0}$. Hence from \eqref{eq:sumcycles}
    \begin{align*}
        \sum_{i=0}^{n-1} w(v_i,v_{i+1}) &\geq s_n (-W) + \eta_{v_0}(\sum_{j=1}^{r_n} |C_j|)\\
        &\geq n \eta_{v_0} - s_n (\eta_{v_0}+W)\\
        \intertext{Dividing by $n$}
        \frac{1}{n} \sum_{i=0}^{n-1} w(v_i,v_{i+1}) &\geq \eta_{v_0} - \frac{s_n}{n}(\eta_{v_0}+W)\\
        \intertext{Letting $n \to \infty$}
        \liminf \left( \frac{1}{n} \sum_{i=0}^{n-1} w(v_i,v_{i+1}) \right) &\geq \liminf \left( \eta_{v_0} - \frac{s_n}{n}(\eta_{v_0}+W) \right) = \eta_{v_0}\\
        \intertext{Hence for any $v_0 v_1 \ldots$ that conforms with $\tau^*$}
        \finf(v_0 v_1 \ldots ) &\geq \eta_{v_0}\\
        \intertext{This shows that}
        \finf(\sigma,\tau^*) &\geq \eta \qquad \forall \sigma \in \ST{0}
    \end{align*}
\end{proof}

In the above proof when $(\sigma,\tau)$ are positional strategies, $(\sigma^*,\tau^*)$ is just $(\sigma,\tau)$. The proof shows that when all the cycles reachable from $v$ in $G_{\sigma}$ have mean weight $\leq v$, then any infinite path $\pi$ from $v$ in $G_{\sigma}$ will have $\fsup(\pi) \leq v$. Similarly if all the cycles reachable from $v$ in $G_\tau$ have mean weight $\geq v$ then any infinite path $\pi$ in $G_{\tau}$ from $v$ will have $\finf(\pi) \geq v$. This shows that if $(\sigma,\tau)$ are postional optimal strategies of $\gsup$ (or $\ginf$), they continue to be optimal strategies for $\G_w$. This argument can also be extended to the case when $(\sigma,\tau)$ are finite memory optimal strategies in $\gsup$ (or $\ginf$) by embedding the memory inside $G$.

\section{Parity to Mean payoff}

The corresponding finite games for Parity and Mean payoff are very similar. This helps establish a reduction from Parity  to Mean payoff games.

Let $G$ be a graph with priorities $p$. Consider the edge weights
\begin{equation}
    \label{eq:parityWeights}
    w(u,v) = -(-|V|)^{p(u)}
\end{equation}
$w$ is defined so that for any simple cycle $C$ in $G$, the max priority in $C$ is odd then $\mean(C)>0$, and if the max priority is even then $\mean(C)<0$. Let $v_r$ be the vertex with the largest priority $C=v_0v_1 \ldots v_{k-1}$. Let $a=w(v_r,v_{r+1})$ (addition is modulo $k$). Notice that any other $w(v_i,v_{i+1})$ either equals $a$ or has absolute value bounded by $\frac{a}{|V|}$. Since $|C| \leq |V|$, $\Sum(C)$ (and hence $\mean(C)$) will have the same sign as $a=w(v_r,v_{r+1})$. When $p(v_r)$ is odd, $a > 0$; when it is even $a < 0$.

Now look at the finite games $(\G^f_p,v)$ and $(\G^f_w,v)$. \Player{1} wins $(\G^f_p,v)$ if and only if \Player{1} can ensure a payoff $>0$ in $(\G^f_w,v)$. If $\eta_v$ is the value of $(\G^f_w, v)$, this is the same condition as $\eta_v > 0$. Pass to their respective infinite games to obtain -- \Player{1} has a winning strategy from $(\G_p,v)$ if and only if $\val(\G_w, v) > 0$.

Hence if we define the following decision problem for the mean payoff game
\begin{decision}[MP]
    \label{dec:meanpayoff}
Given $G$, edge weights $w$ and a $v \in V$, let $\eta_v$ be the value of the game $(\G_w,v)$. Determine whether $\eta_v > 0$ or not.
\end{decision}
Then this gives a reduction from \nameref{dec:parity} to \nameref{dec:meanpayoff}. Notice that we can assume that
\[
    p : V \mapsto \set{0,1 \ldots 2|V|}
\]
Then the edge weights $w$ \eqref{eq:parityWeights} can be constructed in polynomial time. Hence this is a polynomial time reduction.

\section{Summary}

We have introduced Mean payoff games and the concept of optimal value and strategies. Similar to Parity games, an equivalent finite game was used to prove existence of optimal strategies (\autoref{thm:meanEquilibrium}) -- the proof uses the same stack based extension (\autoref{sec:stack}) as used for parity. The complete proof of positional determinacy (\autoref{thm:posMeanPayoff}) can be obtained by doing some more work as presented in \cite{bjorklund_memoryless_2004}. The similarity between the finite games for Parity and Mean payoff is used to give a polynomial time reduction from the decision problem for Parity (\nameref{dec:parity}) to the decision problem for mean payoff (\nameref{dec:meanpayoff}). This brings a new set of techniques to tackle these problems. \cite{zwick_meanpayoff} describes a $O(|V|^3|E|W)$ time algorithm to find the value in the Mean payoff game, however this is only a pseudo polynomial time algorithm because of the linear dependence on $W$.
