\documentclass{beamer}

\usetheme{Rochester}
\usecolortheme{beaver}

\title{Sequential Games}
\author[M. Dewaskar]{Miheer Dewaskar}
\institute[CMI]{Chennai Mathematical Institute}
\subject{Computer Science}
%
% \AtBeginSection[]
% {
%   \begin{frame}
%     \frametitle{Table of Contents}
%     \tableofcontents[currentsection]
%   \end{frame}
% }

\begin{document}
  \frame{\titlepage}

  \begin{frame}
    \frametitle{Table of Contents}
    \tableofcontents
  \end{frame}

  \section{Introduction}
    \subsection{Definition of Parity Games}

    \subsection{An alternative finite game}
    % Shows decidability

    \subsection{Some results}
    \begin{itemize}
      \item Positional winning strategies. Hence in NP $\cap$ Co-NP
      \item In fact, in UP $\cap$ Co-UP. UP is the class of languages having unique a certificate.
      %Interestingly  \cite{up} says that all problems in NP-Co-NP fall in this class too
    \end{itemize}

  \section{Complexity}
    \begin{itemize}
      \item Show to be in UP $\int$ Co-UP but no polynomial time algorithm is known yet
      \item The problem is FPT in the number of colours.
      \item \cite{subexp} gives a subexponential algorithm
    \end{itemize}

  \section{Approaches to solve Parity Games}
    \begin{itemize}
      \item Reduction to Mean-Payoff games
      \item Policy Iteration Methods
      \item Small progress Measures
      \item Recursive approach
    \end{itemize}

  \section{Connection to other areas}
    \begin{itemize}
      \item Mean-payoff games seem like quantitative analogues of Parity Games
      \item Randomized
    \end{itemize}


  \section{References}

    \begin{thebibliography}{1}
      \bibitem{up} Jurdziński, Marcin. “Deciding the Winner in Parity Games Is in UP∩ Co-UP.” Information Processing Letters 68, no. 3 (1998): 119–24.
      \bibitem{subexp} Jurdzinski, Marcin, Mike Paterson, and Uri Zwick. “A Deterministic Subexponential Algorithm for Solving Parity Games.” SIAM Journal on Computing 38, no. 4 (2008): 1519–32.
    \end{thebibliography}
\end{document}
